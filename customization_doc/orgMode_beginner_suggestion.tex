% Created 2017-04-11 Tue 16:26
% Intended LaTeX compiler: pdflatex
\documentclass[11pt]{article}
\usepackage[utf8]{inputenc}
\usepackage[T1]{fontenc}
\usepackage{graphicx}
\usepackage{grffile}
\usepackage{longtable}
\usepackage{wrapfig}
\usepackage{rotating}
\usepackage[normalem]{ulem}
\usepackage{amsmath}
\usepackage{textcomp}
\usepackage{amssymb}
\usepackage{capt-of}
\usepackage{hyperref}
\author{Carsten Dominik}
\date{\today}
\title{Org-Mode Beginners Customization Guide}
\hypersetup{
 pdfauthor={Carsten Dominik},
 pdftitle={Org-Mode Beginners Customization Guide},
 pdfkeywords={},
 pdfsubject={},
 pdfcreator={Emacs 25.1.1 (Org mode 9.0.5)}, 
 pdflang={English}}
\begin{document}

\maketitle
\tableofcontents



\section*{Introduction}
\label{sec:org4bd04a6}
\index{Customization!Introduction}

Org-mode is a highly customizable package.  It currently contains close to
400 customization variables that can be changed to tweak every detail, and
more than 260 are known to be actually used out there.

However, as a beginner you do not care about this kind of flexibility.  On
this page, we have a list of five settings that you might want to try first
in order to personalize your system.

Once you are done with that, we also have a list of some 40 variables that
are \href{http://orgmode.org/worg/org-configs/org-customization-survey.php}{changed by many users}.

\section*{The Basics}
\label{sec:orga88657a}

\index{Customization!Basics}

\subsection*{Minimal customization}
\label{sec:org6de243d}

\index{Customization!Minimal}

The minimal customization needed to use Org-mode is -- \textbf{Nothing at all!}
\par Org-mode works out of the box, and besides the steps described in the
manual to \href{http://orgmode.org/manual/Activation.html\#Activation}{activate} it, \emph{nothing is needed at all}.  Just open a \emph{.org}
file, press \texttt{C-c [} to tell org that this is a file you want to use in your
agenda, and start putting your life into plain text.

OK, for completeness, let's just repeat what is needed to activate
Org-mode in files with \emph{.org} extension, and a few important key
assignments.

\begin{verbatim}
  (add-to-list 'auto-mode-alist '("\\.org\\'" . org-mode))
  (global-set-key "\C-cl" 'org-store-link)
  (global-set-key "\C-ca" 'org-agenda)
  ;; (global-font-lock-mode 1)  Not needed in recent emacsen
\end{verbatim}

\subsection*{Five small steps toward a personalized system}
\label{sec:org9ba8a65}

\subsubsection*{One: More TODO keywords}
\label{sec:org48cf150}

\index{Todo Keywords}

Define the TODO states you find useful and single letters for fast
selection.  Customize the variable \texttt{org-todo-keywords} or simply do this
right in the file with\footnote{press \texttt{C-c C-c} in the line after changing it\label{org4124dac}}:

\begin{verbatim}
  #+TODO: TODO(t) STARTED(s) WAITING(w) | DONE(d) CANCELED(c)
\end{verbatim}

\subsubsection*{Two: Which tags do you use most?}
\label{sec:orge4dbe0d}

\index{Tag}

You can always add tags freely and by hand, but if you configure the
most important ones along with fast-access keys, life will be better.
Configure the variable \texttt{org-tags-alist} or simply do this right in the
file with\textsuperscript{\ref{org4124dac}}

\begin{verbatim}
#+TAGS: home(h) work(w) @computer(c) @phone(p) errants(e)
\end{verbatim}

\subsubsection*{Three: Which files are relevant for the agenda?}
\label{sec:orgc12b8e2}

\index{Agenda}

When Org compiles agenda views like the the agenda for the current
week (\texttt{C-c a a}) or the global TODO list (\texttt{C-c a t}), it checks all
files in the variable \texttt{org-agenda-files}.  Instead of setting this
variable explicitly, it is much easier to just add and remove the
current buffer with \texttt{C-c [} and \texttt{C-c ]}, respectively.

\subsubsection*{Four: Find what you need to see now}
\label{sec:org7eec36f}

If you need to sharpen the predefined agenda commands, define your \href{http://orgmode.org/manual/Custom-agenda-views.html\#Custom-agenda-views}{own}
Agenda commands, using the variable \texttt{org-agenda-custom-commands}.
This is a pretty complex variable, but if you use the customize
interface\footnote{\texttt{M-x customize-variable RET
org-agenda-custom-commands RET}}, it is not too hard.  Everyone
ends up customizing this one after getting comfortable with Org-mode.
Do checkout \href{http://orgmode.org/worg/org-tutorials/org-custom-agenda-commands.php}{this tutorial} on building your own custom agenda commands
as well.

\subsubsection*{Five: Capture ideas with predefined templates}
\label{sec:org8b0dc48}

\index{Capture}

Use \texttt{org-capture} to quickly capture ideas, tasks, and notes.
Populate\footnote{M-x customize-variable RET org-capture-templates RET}
the variable \texttt{org-capture-templates} with templates and target
locations.

\section*{Pretend to be a power-user}
\label{sec:org0385701}

Here is a list of the variables most frequently changed by power
users.  The variables mentioned above are repeated in this list.  Yes,
some power users change 60 options or more, but these showed up most
frequently in the \href{http://orgmode.org/worg/org-configs/org-customization-survey.php}{customization survey}.

\subsubsection*{Basic setup}
\label{sec:org0bd758c}

\index{Directory}
\index{Agenda!Files}
\index{Startup}
\index{Archive!Location}

Here are a few things about basic setup that many people change.

\begin{description}
\item[{org-directory}] Where are my Org files typically located?  Org
acutally uses this variable only under rare circumstances, like
when filing remember notes in an interactive way and prompting
you for an Org file to put the note into.
\end{description}


\begin{description}
\item[{org-agenda-files}] Which \href{http://orgmode.org/manual/Agenda-files.html\#Agenda-files}{files} do I want to be checked for entries
when compiling my agenda?  Many people do not customize this one,
but just use \texttt{C-c [} and \texttt{C-c ]} to add or remove the current
file, respectively.

\item[{org-startup-folded}] When visiting an Org file, in what \href{http://orgmode.org/manual/Visibility-cycling.html\#Visibility-cycling}{folding
state} do I first want to see it?  Many use \texttt{\#+STARTUP} options to
set this on a \href{http://orgmode.org/manual/In\_002dbuffer-settings.html\#In\_002dbuffer-settings}{per-file basis}.

\item[{org-archive-location}] When \href{http://orgmode.org/manual/Archiving.html\#Archiving}{archiving} an entry, where will it go?
\end{description}

\subsubsection*{Editing behavior and appearance}
\label{sec:org4bf9d0e}

\index{Appearance}
\index{Follow links}
\index{Completion}
\index{Ido}
\index{Levels}
\index{Blank}

Besides being an organizer, Org-mode is also a text mode for writing
and taking notes.  The following variables that influence basic
editing behavior and the appearance of the buffer are often
customized:

\begin{description}
\item[{org-hide-leading-stars}] Make the outline more list-like be \href{http://orgmode.org/manual/Clean-view.html\#Clean-view}{hiding}
all leading stars but one.

\item[{org-odd-levels-only}] Should \href{http://orgmode.org/manual/Clean-view.html\#Clean-view}{2 stars} be added per level?  This
makes the indentation more like in a book.

\item[{org-special-ctrl-a/e}] Should \texttt{C-a} and \texttt{C-e} behave specially,
considering the headline and not the leading stars, todo
keywords, or the trailing tags?  About equal numbers of users set
this to \texttt{t} or to \texttt{reversed}

\item[{org-special-ctrl-k}] Should \texttt{C-k} behave \href{http://orgmode.org/worg/org-faq.html\#C-k-is-killing-subtrees}{specially} in headlines,
considering tags and visibility state?

\item[{org-completion-use-ido}] Should \emph{ido.el} be used for completion
whenever it makes sense?

\item[{org-return-follows-link}] Should pressing RET on a hyperlink \href{http://orgmode.org/manual/Handling-links.html\#Handling-links}{follow}
the link?  People who are used to this from web browsers often
make this choice.

\item[{org-blank-before-new-entry}] Org-mode tries to be smart about
inserting blank lines before \href{http://orgmode.org/manual/Structure-editing.html\#Structure-editing}{new entries/items}, by looking at
what is before the previous entry/item.  Customize this to
out-smart it.
\end{description}

\subsubsection*{The TODO keywords}
\label{sec:org86607b9}

\index{Todo Keywords}
\index{Todo Keywords!Faces}
\index{Todo Keywords!Dependencies}
\index{Checkbox}

\begin{description}
\item[{org-todo-keywords}] Which \href{http://orgmode.org/manual/TODO-extensions.html\#TODO-extensions}{TODO keywords} should be used?  Also you
can define keys for \href{http://orgmode.org/manual/Fast-access-to-TODO-states.html\#Fast-access-to-TODO-states}{fast access} here.  Very many people use this,
or define the keywords with a \texttt{\#+TODO:} setting in the buffer.

\item[{org-todo-keyword-faces}] Here you can define different faces for
different TODO keywords.

\item[{org-enforce-todo-dependencies}] Should unfinished children \href{http://orgmode.org/manual/TODO-dependencies.html\#TODO-dependencies}{block}
state changes in the parent?

\item[{org-enforce-todo-checkbox-dependencies}] Should unfinished
checkboxes \href{http://orgmode.org/manual/TODO-dependencies.html\#TODO-dependencies}{block} state changes in the parent?
\end{description}

\subsubsection*{Tags}
\label{sec:org50aa0f7}

\index{Tag}

\begin{description}
\item[{org-tag-alist}] Which \href{http://orgmode.org/manual/Tags.html\#Tags}{tags} should be available?  Note that tags
besides the configured ones can be used, but for the important
ones you can define keys for \href{http://orgmode.org/manual/Setting-tags.html\#Setting-tags}{fast access} here.

\item[{org-tags-column}] How should tags be aligned in the headline?

\item[{org-fast-tag-selection-single-key}] Set this to make the tags
interface even faster, if all you normally do is changing a single
tag.
\end{description}

\subsubsection*{Progress logging}
\label{sec:org30e1e10}

\index{Progress!Logging}
\index{Logging}

\begin{description}
\item[{org-log-done}] Do you want to \href{http://orgmode.org/manual/Progress-logging.html\#Progress-logging}{capture} time stamps and/or notes when
TODO state changes, in particular when a task is DONE?  A simple
setting that many use is \texttt{(setq org-log-done 'time)}.
\end{description}

\subsubsection*{Capture and Refile}
\label{sec:orgfdca395}

\index{Capture}
\index{Refile}

\begin{description}
\item[{org-reverse-note-order}] When adding new entries (or tasks) to a
list, do I want the entry to be first or last in the list?
\end{description}

\texttt{org-capture} is great for fast capture of ideas, notes, and tasks.  It
is one of the primary capture methods in Org-mode.

\begin{description}
\item[{org-capture-templates}] Prepare \href{http://orgmode.org/manual/Remember-templates.html\#Remember-templates}{templates} for the typical notes and
tasks you want to capture quickly.  I believe everyone using
\texttt{org-capture} customizes this.

\item[{org-default-notes-file}] If you do not set up templates with target
files, at least tell Org where to put captured notes.
\end{description}

\emph{Refiling} means moving entries around, for example from a capturing
location to the correct project.

\begin{description}
\item[{org-refile-targets}] What should be on the \href{http://orgmode.org/manual/Refiling-notes.html\#Refiling-notes}{menu} when you refile
tasks with \texttt{C-c C-w}?

\item[{org-refile-use-outline-path}] How would you like to select refile
targets. Headline only, or the path along the outline hierarchy?
\end{description}

\subsubsection*{Agenda Views}
\label{sec:org57dc34e}

\index{Agenda!Views}

\begin{description}
\item[{org-agenda-start-on-weekday}] Should the \href{http://orgmode.org/manual/Weekly\_002fdaily-agenda.html\#Weekly\_002fdaily-agenda}{agenda} start on Monday, or
better today?

\item[{org-agenda-ndays}] How many days should the default agenda show?
Default is 7, a whole week.

\item[{org-agenda-include-diary}] Should the agenda also show \href{http://orgmode.org/manual/Weekly\_002fdaily-agenda.html\#Weekly\_002fdaily-agenda}{entries} from
the Emacs diary?

\item[{org-agenda-custom-commands}] Define your \href{http://orgmode.org/manual/Custom-agenda-views.html\#Custom-agenda-views}{own} Agenda commands.
Complex, advanced variable, but pretty much everyone ends up
configuring it.  Use customize to configure it, this is the best
and safest way.  Do checkout \href{http://orgmode.org/worg/org-tutorials/org-custom-agenda-commands.php}{this tutorial} on building your own
custom agenda commands as well.

\item[{org-agenda-sorting-strategy}] How should things be \href{http://orgmode.org/manual/Sorting-of-agenda-items.html\#Sorting-of-agenda-items}{sorted} in the
agenda display.  Even though I think the defaults are very usable,
power users tend to tweak this.

\item[{org-stuck-projects}] How to find projects that need \href{http://orgmode.org/manual/Stuck-projects.html\#Stuck-projects}{attention}?
\end{description}

To reduce clutter in the task list for today, many users like to
remove tasks from the daily list right when they are done.  The
following variables give detailed control to what kind of entries this
should apply:

\begin{description}
\item[{org-agenda-skip-scheduled-if-done}] Scheduled entries.  Many users
turn this on.

\item[{org-agenda-skip-deadline-if-done }] Deadlines.  Many users turn
this on.

\item[{org-agenda-skip-timestamp-if-done}] Entries with any timestamp,
appointments just like scheduled and deadline entries.
Relatively few users select this one.
\end{description}

People who use Org like a \href{http://www.newartisans.com/2007/08/using-org-mode-as-a-day-planner.html}{day planner}, who \href{http://orgmode.org/manual/Deadlines-and-scheduling.html\#Deadlines-and-scheduling}{schedule} all tasks to
specific dates, often like to not have scheduled tasks listed in their
global TODO list, because scheduling it already means to have taking
care of it in a sense, and because they know they will run into these
tasks in the agenda anyway.

\begin{description}
\item[{org-agenda-todo-ignore-deadlines}] Don't show deadline tasks in
global TODO list.

\item[{org-agenda-todo-ignore-with-date}] Don't show any tasks with a date
in the global TODO list.

\item[{org-agenda-todo-ignore-scheduled}] Don't show scheduled tasks
in the global TODO list.
\end{description}

\subsubsection*{Export/Publishing setup}
\label{sec:org85c5974}

\index{Export}
\index{Publish!Setup}

\begin{description}
\item[{org-export-with-\LaTeX{}-fragments}] Should \href{http://orgmode.org/manual/LaTeX-fragments.html\#LaTeX-fragments}{\LaTeX{} fragments} be
converted to inline images for HTML output?

\item[{org-export-html-style}] Customize the default \href{http://orgmode.org/manual/CSS-support.html\#CSS-support}{style} for HTML
export.

\item[{org-publish-project-alist}] Set up projects that allow many files
to be exported and \href{http://orgmode.org/manual/Publishing.html\#Publishing}{published} with a single command.
\end{description}

\section*{Become a true power user}
\label{sec:orgf1107c0}

If you want to become a true power user, \href{http://thread.gmane.org/gmane.emacs.orgmode/10804}{see} for yourself what some users
do.  The Emacs customization system\footnote{\texttt{M-x org-customize RET}} organizes
all variables into a structure that can be used to easily \href{http://orgmode.org/worg/org-tutorials/org-customize.html}{find the one
particular option} you might be looking for.  Also, the \href{http://orgmode.org/manual/}{Org-mode manual} and
the \href{http://orgmode.org/worg/org-faq.html}{FAQ} mention many variables in the appropriate context.
\end{document}